\documentclass[11pt,a4paper]{article}
\usepackage[latin1]{inputenc}
\usepackage{amsmath}
\usepackage{amsfonts}
\usepackage{amssymb}
\usepackage{graphicx}
\usepackage{titling}
\usepackage{paralist}
\usepackage{tikz}
\usepackage{mathtools}
\usepackage[margin=0.8in]{geometry}
\usepackage[linesnumbered,ruled,vlined]{algorithm2e}

\usepackage{array}
\newcolumntype{L}[1]{>{\raggedright\let\newline\\\arraybackslash\hspace{0pt}}m{#1}}
\newcolumntype{C}[1]{>{\centering\let\newline\\\arraybackslash\hspace{0pt}}m{#1}}
\newcolumntype{R}[1]{>{\raggedleft\let\newline\\\arraybackslash\hspace{0pt}}m{#1}}

\setlength{\droptitle}{-7em}
\posttitle{\par\end{center}}

\newcommand*\xor{\mathbin{\oplus}}
\DeclarePairedDelimiter{\ceil}{\lceil}{\rceil}
\def\doubleunderline#1{\underline{\underline{#1}}}

% Soldity code formating
\usepackage{listings}
\usepackage{xcolor}

\definecolor{verylightgray}{rgb}{.97,.97,.97}
\lstdefinelanguage{Solidity}{
keywords=[1]{anonymous, assembly, assert, balance, break, call, callcode, case, catch, class, constant, continue, constructor, contract, debugger, default, delegatecall, delete, do, else, emit, event, experimental, export, external, false, finally, for, function, gas, if, implements, import, in, indexed, instanceof, interface, internal, is, length, library, log0, log1, log2, log3, log4, memory, modifier, new, payable, pragma, private, protected, public, pure, push, require, return, returns, revert, selfdestruct, send, solidity, storage, struct, suicide, super, switch, then, this, throw, transfer, true, try, typeof, using, view, while, with, addmod, ecrecover, keccak256, mulmod, ripemd160, sha256, sha3}, % generic keywords including crypto operations
keywordstyle=[1]\color{blue}\bfseries,
keywords=[2]{address, bool, byte, bytes, bytes1, bytes2, bytes3, bytes4, bytes5, bytes6, bytes7, bytes8, bytes9, bytes10, bytes11, bytes12, bytes13, bytes14, bytes15, bytes16, bytes17, bytes18, bytes19, bytes20, bytes21, bytes22, bytes23, bytes24, bytes25, bytes26, bytes27, bytes28, bytes29, bytes30, bytes31, bytes32, enum, int, int8, int16, int24, int32, int40, int48, int56, int64, int72, int80, int88, int96, int104, int112, int120, int128, int136, int144, int152, int160, int168, int176, int184, int192, int200, int208, int216, int224, int232, int240, int248, int256, mapping, string, uint, uint8, uint16, uint24, uint32, uint40, uint48, uint56, uint64, uint72, uint80, uint88, uint96, uint104, uint112, uint120, uint128, uint136, uint144, uint152, uint160, uint168, uint176, uint184, uint192, uint200, uint208, uint216, uint224, uint232, uint240, uint248, uint256, var, void, ether, finney, szabo, wei, days, hours, minutes, seconds, weeks, years},    % types; money and time units
keywordstyle=[2]\color{teal}\bfseries,
keywords=[3]{block, blockhash, coinbase, difficulty, gaslimit, number, timestamp, msg, data, gas, sender, sig, value, now, tx, gasprice, origin},    % environment variables
keywordstyle=[3]\color{violet}\bfseries,
identifierstyle=\color{black},
sensitive=false,
comment=[l]{//},
morecomment=[s]{/*}{*/},
commentstyle=\color{gray}\ttfamily,
stringstyle=\color{red}\ttfamily,
morestring=[b]',
morestring=[b]"
}
\lstset{
language=Solidity,
backgroundcolor=\color{verylightgray},
extendedchars=true,
basicstyle=\footnotesize\ttfamily,
showstringspaces=false,
showspaces=false,
numbers=left,
numberstyle=\footnotesize,
numbersep=9pt,
tabsize=2,
breaklines=true,
showtabs=false,
captionpos=b
}

\title{Static taint analysis for Ethereum contracts \\ \large Program analysis for system security and reliability}
\author{Eric Marty, Yannick Merkli}
\date{\today}

\begin{document}

\maketitle
\pagenumbering{arabic}
\setlength\parindent{0pt}

In this project we implement a static taint analyzer for Ethereum smart contracts written in Solidity.
We want to label contracts as \textbf{Safe} or \textbf{Tainted} based on whether an untrusted user can call \lstinline{selfdestruct} and therefore remove the contract from the blockchain.

More concretely we label a contract as \textbf{Tainted} if \lstinline{msg.sender} is tainted when \lstinline{selfdestruct} is invoked.
Taints arise from function and \lstinline{msg} arguments.
A taint can be removed by a sanitizer, a branch (\lstinline{if}, \lstinline{require}, \lstinline{assert}) depending on a guard.
A guard is a statement, explicitly depending on \lstinline{msg.sender} and all other values it depends must not be tainted.

\section{Analyzer Structure}
The solidity contracts are parsed and translated into a given Intermediate Representation (IR) and then analyzed in Datalog.

\subsection{Function Context \& Block Graph}
To catch contextual relationships between functions, we implement \lstinline{contexts}, based on a call stack, that allows us to assign taint and sanitized tags to statements, based on their function context.
We initially build a call stack for each public function in the contract (\lstinline{contexts.contextForInit}) and then recursively extend those contexts for internal function calls (\lstinline{contexts.contextForCall}), up to a limit of 3 recursive function calls, to prevent infinite recursion.

\lstinputlisting[language=Solidity]{example_contracts/context.sol}

We also build a \lstinline{block_graph}, that represents the program flow inside of functions of a contract.
This block graph provides a \lstinline{dominatedBy} rule, that allows us to easily identify which blocks are always traversed to get to other blocks.

\subsection{Taint}
We implement two kinds of taint tags.
We distinguish between hard taints (\lstinline{is_tainted}) and weak taints (\lstinline{maybe_tainted}) and their counterparts for storage variables \lstinline{tainted_storage} and \lstinline{maybe_tainted_storage}.
Taint tags are propagated in the code on a per statement basis and are passed into and out of internal function calls.

We assign a hard taint tag to all function arguments in top level functions, any occurrences of \lstinline{msg.sender} and \lstinline{msg.value}, loads from tainted storage variables and return values from functions of context depth $>3$.
We assign a weak taint tag to elements, if they load from a weak tainted storage field.

\lstinputlisting[language=Solidity]{example_contracts/taints.sol}

We assign storage fields a hard/weak taint tag if they store a value that has a hard/weak taint assigned to it.
Additionally we assign storage fields a weak taint tag on the first line in every top level function, if there exists a function where a storage field has a weak/hard taint tag assigned on the last line.

\lstinputlisting[language=Solidity]{example_contracts/storage.sol}

\subsection{Sanitizer}

We treat every guard condition that depends on \lstinline{msg.sender} as a potential guard (\lstinline{maybe_guard}), however we do not immediately check for taint tags on the other elements.
Instead, we build a stack of \lstinline{(SSA, Context)} tuples, which includes each statement the guard depends on (except \lstinline{msg.sender}) and the context in which it was assigned.
This is done in order to avoid cyclic redundancies between \lstinline{sanitized} and \lstinline{tainted} tags.
A block which branches on a \lstinline{maybe_guard} (\lstinline{maybeGuardBlock}) forward propagates \lstinline{maybeSantitizedBlock} tags into descendant blocks and descendant function calls.
It further backpropagates \lstinline{maybeSantitizedBlock} tags to all blocks of equal or lower branching depth.

We further handle functions with multiple returns as an edge case, since multiple returns allow for guard evasion.
We assign \lstinline{notSanitizedBlock} tags to blocks, to which multiple returns point to, of which at least one has no forward propagated \lstinline{maybeSantitizedBlock} tag.

\lstinputlisting[language=Solidity]{example_contracts/multi_return.sol}

\subsection{Tying Everything Together}
A sink (\lstinline{selfdestruct}) is tainted if the block it lies in is not a forward propagated \lstinline{maybeSantitizedBlock}, the block is a forward propagated \lstinline{maybeSantitizedBlock} but at least one program execution path does not pass a guard (\lstinline{notSanitizedBlock}) or the block is a forward propagated \lstinline{maybeSantitizedBlock} but the SSA stack of its guard, contains at least one tuple \lstinline{(id, ctx)} where \lstinline{id} is tainted in context \lstinline{ctx}.

\end{document}
